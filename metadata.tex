% DO NOT EDIT - automatically generated from metadata.yaml

\def \codeURL{https://github.com/tawantayron/RE-Fear-and-extinction-in-a-BA-net-model}
\def \codeDOI{}
\def \codeSWH{}
\def \dataURL{}
\def \dataDOI{}
\def \editorNAME{}
\def \editorORCID{}
\def \reviewerINAME{}
\def \reviewerIORCID{}
\def \reviewerIINAME{}
\def \reviewerIIORCID{}
\def \dateRECEIVED{01 March 2021}
\def \dateACCEPTED{}
\def \datePUBLISHED{}
\def \articleTITLE{[Re] Context-Dependent Encoding of Fear and Extinction Memories in a Large-Scale Network Model of the Basal Amygdala}
\def \articleTYPE{Replication / Computational Neuroscience}
\def \articleDOMAIN{}
\def \articleBIBLIOGRAPHY{bibliography.bib}
\def \articleYEAR{2021}
\def \reviewURL{}
\def \articleABSTRACT{The basal nucleus of the amygdala (BA) is related to the process of creating memories of conditioned fear and extinction that are both dependent on the context. Vlacho and collaborators developed two models of neural networks to study the effect of plasticity based on a specific phenomenological rule in BA excitatory neurons. When an excitatory subpopulation receives conditioned stimulus (CS) and contextual inputs in narrow time windows, synaptic weights are potentiated by this effect, increasing the subpopulation's firing rate related to the specified context. In this replication, we implemented the models using Python (for the mean-field model) and Brian 2 (for the spiking neuron model), and we were able to reproduce the original results qualitatively. In order to replicate the model, it was necessary to estimate a considerable amount of parameters and to adapt some of the protocols that were either ambiguous or absent in the methodological descriptions of the original work.}
\def \replicationCITE{}
\def \replicationBIB{}
\def \replicationURL{}
\def \replicationDOI{}
\def \contactNAME{Tawan T. A. Carvalho}
\def \contactEMAIL{tawantayron@yahoo.com.br}
\def \articleKEYWORDS{Basal amygdala, Fear and extinction memories, Context-dependent, Conditioned stimulus, Spiking network model}
\def \journalNAME{ReScience C}
\def \journalVOLUME{4}
\def \journalISSUE{1}
\def \articleNUMBER{}
\def \articleDOI{}
\def \authorsFULL{Tawan Tayron Andrade de Carvalho and Luana Barreto Domingos and Renan Oliveira Shimoura and Nilton Liuji Kamiji and Vinicius Lima Cordeiro and Mauro Copelli and Antonio C. Roque}
\def \authorsABBRV{T. T. A. Carvalho and L.B. Domingos and R. Shimoura and N.L. Kamiji and V.L. Cordeiro and M. Copelli and A. C. Roque}
\def \authorsSHORT{Carvalho et al}
\title{\articleTITLE}
\date{}
\author[1,\orcid{0000-0001-9583-4830}]{Tawan T. A. Carvalho}
\author[2,\orcid{0000-0002-7022-6502}]{Luana B. Domingos}
\author[3,\orcid{0000-0002-6580-5999}]{Renan O. Shimoura}
\author[3,\orcid{0000-0001-5006-6612}]{Nilton L. Kamiji}
\author[4,\orcid{0000-0001-7115-9041}]{Vinicius L. Cordeiro}
\author[1,\orcid{0000-0001-7441-2858}]{Mauro Copelli}
\author[3,\orcid{0000-0003-1260-4840}]{Antonio C. Roque}
\affil[1]{Department of Physics, Federal University of Pernambuco, Recife, PE, Brazil}
\affil[2]{Department of Pharmacology, Ribeir\~{a}o Preto Medical School (FMRP), University of S\~{a}o Paulo, Ribeir\~{a}o Preto, SP, Brazil}
\affil[3]{Department of Physics, School of Philosophy, Sciences and Letters of Ribeir\~{a}o Preto (FFCLRP), University of S\~{a}o Paulo, Ribeir\~{a}o Preto, SP, Brazil}
\affil[4]{Systems Neuroscience Institut (INS, UMR 1106), Aix-Marseille University, Marseille, France}

